 \hoffset1cm %%% przesunięcie poziome (przykładowo) o 1cm przeznaczone na oprawę
 \documentclass[11pt,a4paper]{report}

\usepackage[utf8]{inputenc}
\usepackage{polski}
 
 \usepackage{amsmath}
 \usepackage{amssymb}
 \usepackage{indentfirst}
 \pagestyle{headings}
 \renewcommand{\chaptername}{Rozdział}
 \renewcommand{\contentsname}{Spis treści}
 \renewcommand{\bibname}{Bibliografia}

 \begin{document}

 \begin{titlepage}
 \center{\large\scshape Politechnika Krakowska \\
         \normalsize im. Tadeusza Kościuszki}
 \center{\scshape Wydział Inżynierii Elektrycznej i Komputerowej\\
         Kierunek Informatyka}
 \vspace{0.1\textheight}
 \center{\scshape Michał Patyk}
 \bigskip
 \center{\LARGE\bfseries Sterownik pieca kominkowego, w oprarciu o mikrokontroler (lub platformę komputerową), zgodny ze szkicem specyfikacji Wep Thing API}
 \center{(praca licencjacka)}
 \vspace{0.3\textheight}
 \par
 \rightline{Promotor: Dr Radosław Czarnecki}

 \vspace{0.1\textheight}
 \center{Kraków 2019}
 \end{titlepage}

 \tableofcontents

 \chapter*{Wstęp}
 \addcontentsline{toc}{chapter}{Wstęp}

 Głównym celem pracy jest \ldots
 \par
 Praca ta opiera się na wiadomościach zaczerpniętych między innymi
 z~podręcznika \cite{panek}.

 \chapter{Cele pracy, zakres pracy, założenia}

 \section{Cele pracy}
 Celem niniejszej pracy jest opracowanie sterownika pieca kominkowego zgodnego ze szkicem specyfikacji Web Thing API.
 
 \section{Zakres pracy}
 Zakres pracy obejmuje:
 \begin{enumerate}
 \item rozeznanie wśród istniejących rozwiązań - zarówno sprzętowych jak i programowych; komputerowe systemy sterowania, inżynieria systemów informacyjnnch
 \item opracowanie koncepcji; komputerowe systemy sterowania, mikroprocesory i mikrokontrolery, systemy operacyjne, 
 \item wybór podzespołów; architektura systemów komputerowych, podstawy elektroniki i techniki cyfrowej, mikroprocesory i mikrokontrolery, systemy wbudowane
 \item wykonanie prototypu na płytce stykowej; systemy wbudowane
 \item stworzenie oprogramowania; symulacja komputerowa, systemy wbudowane, technologie obiektowe, programowanie obiektowe, sieci komputerowe
 \item przetestowanie oprogramowania; systemy odporne na błędy
 \item zaprojektowanie płytki drukowanej - PCB; elektrotechnika, podstawy elektroniki i techniki cyfrowej
 \item zaprojektowanie obudowy
 \item zintegrowanie z Mozilla Gateway
 \end{enumerate}
 
 \section{Założenia i wymagania}
 
 Wykorzystane narzędzia:
 \begin{enumerate}
 \item[•] środowisko programistyczne CLion
 \item[•] ekosystem PlatformIO
 \item[•] oprogramowanie do projektowania PCB KiCad
 \item[•] platforma monitoringu i kontroli urządzeń WebThing Mozilla
 \end{enumerate}
 
 Sterownik ma umożliwić:
 \begin{enumerate}
 \item[•] monitorowanie pracy pieca kominkowego
 \item[•] lokalne i zdalne zadawanie temperatur
 \item[•] informowanie o zdarzeniach
 \end{enumerate}
 
 Wymagania:
 \begin{enumerate}
 \item[•] zgodność z Web Thing API
 \item[•] watchdog
 \item[•] możliwość rozbudowy o dodatkowe czujniki i elementy wykonawcze
 \end{enumerate}
 
  
 
 \section{Efekt końcowy}
 Planowanym efektem końcowym pracy będzie stworzenie sterownika pieca kominkowego, pozwalającego na bezobsługową pracę paleniska pomiędzy momentami uzupełniania paliwa.

 Element wyróżniający wykonany sterownik stanowi wykorzystanie Web Thing REST API, który pozwala na wykorzystanie sieci jako zunifikowanej warstwy abstarakcji dla zdecentralizowanego internetu rzeczy.
 


 \chapter{Tytuł drugiego rozdziału}
 W tym rozdziale wykorzystamy pojęcie funkcjonału, które
 wprowadziliśmy na stronie~\pageref{poj:funkcjonal}.

 \chapter*{Podsumowanie}
 \addcontentsline{toc}{chapter}{Podsumowanie}

 Nasz wybór (patrz: s.~\pageref{sec:wybor}) miał istotne znaczenie.
 O tym, co było następnie, pisaliśmy w podrozdziale \ref{sec:nastepnie}.
 Konsekwencje\,\ldots

 \begin{thebibliography}{9}
 \addcontentsline{toc}{chapter}{Spis literatury}
 \bibitem{panek}
 E. Panek, \emph{Ekonomia matematyczna}, Akademia Ekonomiczna w Poznaniu,
 Poznań 2003.
 \bibitem{kolejnapozycja}
 \ldots
 \end{thebibliography}

 \end{document}
