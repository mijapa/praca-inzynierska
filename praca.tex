 \hoffset1cm %%% przesunięcie poziome (przykładowo) o 1cm przeznaczone na oprawę
 \documentclass[11pt,a4paper]{report}

\usepackage[utf8]{inputenc}
\usepackage{polski}
 
 \usepackage{amsmath}
 \usepackage{amssymb}
 \usepackage{indentfirst}
 \pagestyle{headings}
 \renewcommand{\chaptername}{Rozdział}
 \renewcommand{\contentsname}{Spis treści}
 \renewcommand{\bibname}{Bibliografia}

 \begin{document}

 \begin{titlepage}
 \center{\large\scshape Politechnika Krakowska \\
         \normalsize im. Tadeusza Kościuszki}
 \center{\scshape Wydział Inżynierii Elektrycznej i Komputerowej\\
         Kierunek Informatyka}
 \vspace{0.1\textheight}
 \center{\scshape Michał Patyk}
 \bigskip
 \center{\LARGE\bfseries Sterownik pieca kominkowego, w oprarciu o mikrokontroler (lub platformę komputerową), zgodny ze szkicem specyfikacji Wep Thing API}
 \center{(praca licencjacka)}
 \vspace{0.3\textheight}
 \par
 \rightline{Promotor: Dr Radosław Czarnecki}

 \vspace{0.1\textheight}
 \center{Kraków 2019}
 \end{titlepage}

 \tableofcontents

 \chapter*{Wstęp}
 \addcontentsline{toc}{chapter}{Wstęp}

 Głównym celem pracy jest \ldots
 \par
 Praca ta opiera się na wiadomościach zaczerpniętych między innymi
 z~podręcznika \cite{panek}.

 \chapter{Tytuł pierwszego rozdziału}
 Poniżej opisujemy \dots

 \section{Tytuł pierwszego podrozdziału}\label{sec:wybor}
 Wybraliśmy tak:
 \begin{itemize}
 \item
 obiekt pierwszy,
 \item
 obiekt drugi,
 \item
 ... itd.
 \end{itemize}
 Teraz zajmiemy się pojęciem \emph{funkcjonału}.\label{poj:funkcjonal}
 \section{Tytuł drugiego podrozdziału}\label{sec:nastepnie}
 Następnie \ldots


 \chapter{Tytuł drugiego rozdziału}
 W tym rozdziale wykorzystamy pojęcie funkcjonału, które
 wprowadziliśmy na stronie~\pageref{poj:funkcjonal}.

 \chapter*{Podsumowanie}
 \addcontentsline{toc}{chapter}{Podsumowanie}

 Nasz wybór (patrz: s.~\pageref{sec:wybor}) miał istotne znaczenie.
 O tym, co było następnie, pisaliśmy w podrozdziale \ref{sec:nastepnie}.
 Konsekwencje\,\ldots

 \begin{thebibliography}{9}
 \addcontentsline{toc}{chapter}{Spis literatury}
 \bibitem{panek}
 E. Panek, \emph{Ekonomia matematyczna}, Akademia Ekonomiczna w Poznaniu,
 Poznań 2003.
 \bibitem{kolejnapozycja}
 \ldots
 \end{thebibliography}

 \end{document}
